%Class and Margins
%--------------------------------------
\documentclass[12pt,a4paper]{report}
\usepackage[left=2.54cm, right=2.54cm, top=1.9cm, bottom=1.9cm]{geometry}
%--------------------------------------

%Graphics
%--------------------------------------
\usepackage{graphicx}
\graphicspath{{./figures/}}
%--------------------------------------

%Encoding
%--------------------------------------
\usepackage[utf8]{inputenc}
\usepackage[T1]{fontenc}
%--------------------------------------
 
%Portuguese-specific commands
%--------------------------------------
\usepackage[portuguese]{babel}
%--------------------------------------
 
%Hyphenation rules
%--------------------------------------
%\usepackage{hyphenat}
%\hyphenation{mate-mática recu-perar}
%--------------------------------------

\newcommand\HRule{\rule{\textwidth}{1pt}}
\renewcommand*\rmdefault{ptm}


%encoding
%--------------------------------------
\usepackage[utf8]{inputenc}
\usepackage[T1]{fontenc}
%--------------------------------------
 
%Portuguese-specific commands
%--------------------------------------
\usepackage[portuguese]{babel}
%--------------------------------------
 
%Hyphenation rules
%--------------------------------------
\usepackage{hyphenat}
\hyphenation{mate-mática recu-perar}
%--------------------------------------




\begin{document}

% 1st page
\begin{titlepage}
    \begin{center}
        
        \vspace*{1cm}
        
        \textbf{Processmanto de Sinais Biomédicos}
        
        \vspace{0.5cm}
        Processamento de Sinal e Imagem Biomédica I
        
        \vspace{1.5cm}
        
        \textbf{10199 - André Cruz}
        
        \vfill
        
        A thesis presented for the degree of\\
        Doctor of Philosophy
        
        \vspace{0.8cm}
        
        \includegraphics[width=0.4\textwidth]{logoipca}
        
        Escola Superior de Tecnologia\\
        Instituto Politécnico do Cávado e Ave\\
        Barcelos, 27 de dezembro de 2017
        
    \end{center}
\end{titlepage}


% intro page
\newpage

\section{Introdução}
This is the first section.\\
Lorem  ipsum  dolor  sit  amet,  consectetuer  adipiscing  
elit.   Etiam  lobortisfacilisis sem.  Nullam nec mi et 
neque pharetra sollicitudin.  Praesent imperdietmi nec ante. 
Donec ullamcorper, felis non sodales...

\subsection{Resumo da Introdução}

% cont page
\newpage

\section{Nivel 1}
This is the first section.

Lorem  ipsum  dolor  sit  amet,  consectetuer  adipiscing  
elit.   Etiam  lobortisfacilisis sem.  Nullam nec mi et 
neque pharetra sollicitudin.  Praesent imperdietmi nec ante. 
Donec ullamcorper, felis non sodales...

\subsection{Nivel 2}

\begin{wrapfigure}{l}{0.5\textwidth}
  \begin{center}
    	\includegraphics[width=0.30\textwidth]{university}
  \end{center}
  \caption{Logo da Universidade 2}
\end{wrapfigure}

This is the first section.

Lorem  ipsum  dolor  sit  amet,  consectetuer  adipiscing  
elit.   Etiam  lobortisfacilisis sem.  Nullam nec mi et 
neque pharetra sollicitudin.  Praesent imperdietmi nec ante. 
Donec ullamcorper, felis non sodales...

This is the first section.

Lorem  ipsum  dolor  sit  amet,  consectetuer  adipiscing  
elit.   Etiam  lobortisfacilisis sem.  Nullam nec mi et 
neque pharetra sollicitudin.  Praesent imperdietmi nec ante. 
Donec ullamcorper, felis non sodales...


This is the first section.

Lorem  ipsum  dolor  sit  amet,  consectetuer  adipiscing  
elit.   Etiam  lobortisfacilisis sem.  Nullam nec mi et 
neque pharetra sollicitudin.  Praesent imperdietmi nec ante. 
Donec ullamcorper, felis non sodales...







\end{document}


% \input{filename} imports the commands from filename.tex into the target file; it's equivalent to typing all the commands from filename.tex right into the current file where the \input line is.

% \include{filename} essentially does a \clearpage before and after \input{filename}, together with some magic to switch to another .aux file, and omits the inclusion at all if you have an \includeonly without the filename in the argument. This is primarily useful when you have a big project on a slow computer; changing one of the include targets won't force you to regenerate the outputs of all the rest.

% \include{filename} gets you the speed bonus, but it also can't be nested, can't appear in the preamble, and forces page breaks around the included text.

